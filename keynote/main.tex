\documentclass[spanish]{beamer}
%
\include{conf/preconfig}
\include{conf/packages}
\include{conf/config}
\include{beamerconf}
%

%\usetheme{Warsaw}
\usetheme{Marburg}
%\usetheme{Antibes}
%\usetheme{Berlin}
%\usetheme{classic}
%\usetheme{Darmstadt}
%\usetheme{Montpellier}
%\usetheme{Goettingen}
\usecolortheme{orchid}
%
%
%\maketitle

\title{Cálculo de flujo subterráneo generado por bombeo}
\author{Christian N. Pfarher, Juan Pablo Garbarino, Marina Castro\\
\textit{Trabajo práctico final de ``Métodos numéricos y simulación'', II-FICH-UNL.}}
\markboth{Método numérico y simulación: TRABAJO FINAL}{}
\date{\today}
%
\begin{document}
%
\frame{\titlepage}
%
%%%%%%%%%%%%%%%%%%%%%%%%%%%%%%%%%%%%%%%%%%%%%%%%%%%%%%%%%
\section{Objetivo}
\begin{frame}
  \frametitle{Objetivo}
Simular el flujo, en un medio poroso (acuífero confinado, homogéneo e isótropo), generado por un par de bombas de extracción de agua en un campo de bombeo y comprobar la existencia o no de algún tipo de interacción entre los pozos de succión de agua. 
\end{frame}
%
%%%%%%%%%%%%%%%%%%%%%%%%%%%%%%%%%%%%%%%%%%%%%%%%%%%%%%%%%
\section{Introducción}
\subsection{Acuífero libre o no confinado}
\begin{frame}{Acuífero Libre o no confinado}
\begin{center}
\includegraphics[width=7cm]{../img/libre}
\end{center}
\end{frame}
%  \begin{itemize}
%  \item<1-> Transformada de Hough
%  \item<2-> Histograma
%  \end{itemize}

\subsection{Acuífero confinado}
\begin{frame}{Acuífero Confinado}
\begin{center}
\includegraphics[width=7cm]{../img/confinado}
\end{center}
\end{frame}
%\begin{frame}{frame confiando}
%  \begin{itemize}
%  \item<1-> Transformada de Hough
%  \item<2-> Histograma
%  \end{itemize}
%\end{frame}
%
%%%%%%%%%%%%%%%%%%%%%%%%%%%%%%%%%%%%%%%%%%%%%%%%%%%%%%%%%
\section{Base Teórica}
%
\subsection{Ec. de Navier-Stokes}
\begin{frame}{Ecuación de Navier-Stokes}
%\begin{center}
%  \includegraphics[width=9cm]{../img/confinado}
%\end{center}
%\begin{itemize}
%\item Umbralizado:
%  \begin{equation*}
%    \label{umbral}
%    f(I)=
%    \begin{cases}
%      0, & I\leq U\\
%      255, & I > U
%    \end{cases}
%  \end{equation*}
%\item Salida: vector de 60 características
%\end{itemize}
\end{frame}
%
\subsection{Ley de Darcy}
\begin{frame}{Técnica 2: Estadísticas del histograma}
%  \includegraphics[width=7cm]{../img/confinado}
%\end{frame}
%\begin{frame}{Técnica 2: Estadísticas del histograma}
%  \includegraphics[width=10cm]{../img/confinado}
%  \begin{itemize}
%  \item Salida: vector de 45 características
%  \end{itemize}
\end{frame}
%
%%%%%%%%%%%%%%%%%%%%%%%%%%%%%%%%%%%%%%%%%%%%%%%%%%%%%%%%%
\section{Desarrollo del problema}
%
\subsection{Datos del problema}
\subsubsection{Geometría G1}
\begin{frame}{Entrenamiento}
%  \begin{itemize}
%  %% \item Extracción de características: a partir de cada imagen junto a su etiqueta
%  %% \item Etiquetado de imágenes: asignación de nombre a cada imagen
%  %% \item Generación de prototipos: promediado de caracteristicas de las 10
%  %%   imágenes de cada etiqueta
%  \item Entrada: base de datos de imágenes etiquetadas
%  \item Se generan ``prototipos'' que promedian las características de las
%    imágenes con igual etiqueta
%  \end{itemize}
\end{frame}
%
\subsubsection{Geometría G2}
\begin{frame}{Clasificación}
%  \begin{itemize}
%  \item Entrada: imagen a clasificar
%  \item Se calculan las características de la imagen a clasificar
%  \item Se etiqueta según el prototipo que minimice el MSE entre sus
%    características y las de la imagen:
%\begin{equation}
%  p_{\T{ganador}}=\arg \min_i\left\{ \frac{1}{\sum N_j}
%                \sum_{j=1}^K\sum_{n=1}^{N_j}(t_j[n]-p_{ij}[n])^2\right\}
%\end{equation}
%\item Donde $p_i$ será el prototipo ganador
%\item Con la etiqueta del prototipo, se encuentra la clase
%  \end{itemize}
\end{frame}
%
\subsection{Relación Caudal - Velocidad}
\begin{frame}{algo}
\end{frame}
%%%%%%%%%%%%%%%%%%%%%%%%%%%%%%%%%%%%%%%%%%%%%%%%%%%%%%%%%
\subsection{Propiedades del medio y de los Materiales}
\subsubsection{Material Suelo}
\subsubsection{Material Wall o Pared}
%%%%%%%%%%%%%%%%%%%%%%%%%%%%%%%%%%%%%%%%%%%%%%%%%%%%%%%%%
\subsection{Condiciones de Borde}
\subsubsection{Fijar Velocidad}
\subsubsection{Fijar Presión}
%%%%%%%%%%%%%%%%%%%%%%%%%%%%%%%%%%%%%%%%%%%%%%%%%%%%%%%%%
\subsection{Mallado}
%%%%%%%%%%%%%%%%%%%%%%%%%%%%%%%%%%%%%%%%%%%%%%%%%%%%%%%%%
\subsection{Ejecución}
%%%%%%%%%%%%%%%%%%%%%%%%%%%%%%%%%%%%%%%%%%%%%%%%%%%%%%%%%
\section{Resultados}
%
\subsection{Resultados para G1}
\begin{frame}{Armado de la base de datos}
%  \begin{itemize}
%  \item Imágenes de 640x480
%  \item Obtenidas con celular
%  \item Diferentes condiciones de iluminación: día, noche, interiores
%  \end{itemize}
\end{frame}
\subsection{Resultados para G2}
%
%\subsection{Conjunto de Imágenes}
%\begin{frame}{Base de datos}
%  \includegraphics[width=10cm]{../img/confinado}
%\end{frame}
%%
%\begin{frame}{Imágenes de prueba (validación)}
%  \includegraphics[width=10cm]{../img/confinado}
%\end{frame}
%
%\subsection{Sólo con técnica de T. de Hough}
\begin{frame}{Obtención de parámetros óptimos para Hough}
%  Determinación de umbral y cantidad de máximos a utilizar
%  \begin{center}
%    \includegraphics[width=9cm]{../img/confinado}
%  \end{center}
\end{frame}
%
%\subsection{Sólo con técnica de histograma}
\begin{frame}{Prueba del método}
%Se generan 3 particiones con 5 etiquetas c/u y se prueba:
%  \begin{itemize}
%  \item Utilizando la técnica de Hough sola
%  \item Utilizando la técnica de histogramas
%  \item Utilizando Hough e histogramas con igual ponderación
%  \end{itemize}
%También se prueba para la base de datos completa
\end{frame}
%
%\subsection{Cómo se hicieron?}
\begin{frame}{Procedimiento}
%  \begin{center}
%    \includegraphics[width=10cm]{../img/confinado}
%  \end{center}
\end{frame}
%
%%%%%%%%%%%%%%%%%%%%%%%%%%%%%%%%%%%%%%%%%%%%%%%%%%%%%%%%%
\section{Resultados}
\begin{frame}{Resultados}
%  Se considera la tasa de error según:
%  \begin{equation*}
%    E_{\%}=100\cdot\frac{\T{número de errores}}{\T{número de pruebas}},
%  \end{equation*}
%
%  Tasas de error para las técnicas de extracción de características
%  \begin{center}\begin{tabular}{ccc}
%      \hline \emph{{Técnica}} & \emph{5 etiquetas} & \emph{15 etiquetas}\\
%      \hline Histogramas & 0\% & 0\%\\
%      \hline Hough & 35.5\% & 60.43\%\\
%      \hline Ambas & 2.22\% & 4.17\%\\
%      \hline
%  \end{tabular}\end{center}
\end{frame}
%
%%%%%%%%%%%%%%%%%%%%%%%%%%%%%%%%%%%%%%%%%%%%%%%%%%%%%%%%%
\section{Conclusiones}
\begin{frame}{Conclusiones}
%  \begin{itemize}
%  \item Satisfactorio coinciderando restricciones
%  \item Optimización para dispositivos móviles
%  \item Preprocesamiento de la imágen
%  \end{itemize}
\end{frame}
%
%%%%%%%%%%%%%%%%%%%%%%%%%%%%%%%%%%%%%%%%%%%%%%%%%%%%%%%%%
\section{Trabajos futuros}
\begin{frame}{Trabajos futuros}
%  \begin{itemize}
%  \item Preprocesamiento
%  \item Filtrado homomórfico
%  \item Warping
%  \item Optimización de la implementación
%  \item Hough local
%  \end{itemize}
\end{frame}
%%%%%%%%%%%%%%%%%%%%%%%%%%%%%%%%%%%%%%%%%%%%%%%%%%%%%%%%%
\begin{frame}{}
%  \begin{itemize}
%  \item<1-1> ¿Preguntas?
%  \item<2-2> ¡Muchas gracias!
%  \end{itemize}
\end{frame}
\end{document}
